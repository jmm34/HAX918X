\documentclass[notes,professionalfont,11pt,usenames,dvipsnames]{beamer}

\usepackage[utf8]{inputenc}
\usepackage[T1]{fontenc}

\usepackage{pxfonts}
\usepackage{eulervm}

\usepackage{pgf,pgfarrows,pgfnodes,pgfautomata,pgfheaps,pgfshade}
\usepackage{tikz}
\usetikzlibrary{plothandlers,plotmarks,arrows,automata}
\usepackage{amsmath,amssymb}
\usepackage{colortbl}
\usepackage{mathrsfs}
\renewcommand{\mathcal}{\mathscr}
\usepackage{enumerate}

\newcommand\independent{\protect\mathpalette{\protect\independenT}{\perp}}
\def\independenT#1#2{\mathrel{\rlap{$#1#2$}\mkern2mu{#1#2}}}

\newcommand{\disp}{\displaystyle}
\newcommand{\ds}{\displaystyle}
\newcommand{\vs}{\bigskip}
\newcommand{\esp}{\mathbb E}
\newcommand{\prob}{\mathbb P}
\newcommand{\ee}{\mathrm e}
\newcommand{\ii}{\mathrm i}
\newcommand{\dd}{\text{d}}
\newcommand{\Int}{\mathrm{int}\,}
\newcommand{\Cl}{\mathrm{adh}\,}
\newcommand{\Supp}{\operatorname{Supp}}
\newcommand{\Card}{\operatorname{Card}}
\newcommand{\Var}{\operatorname{Var}}
\newcommand{\Ave}{\operatorname{Ave}}
\renewcommand{\mathcal}{\mathscr}

\newcommand{\logit}{\operatorname{logit}}
\renewcommand{\epsilon}{\varepsilon}
\newcommand{\eps}{\varepsilon}
\newfont{\manfnt}{manfnt}
\newcommand{\danger}{{\manfnt\symbol{'177}}}
\newcommand{\mdanger}{\marginpar[\hfill\danger]{\danger\hfill}}
\newcommand{\new}{{\manfnt\symbol{30}}}
\newcommand{\mnew}{\marginpar[\hfill\new]{\hfill\new}}
\renewcommand{\P}{\mathbb{P}}
\newcommand{\E}{\mathbb{E}}
\newcommand{\V}{\mathbb{V}}
\newcommand{\tbrown}[1]{\textcolor{brown}{#1}}
\DeclareTextFontCommand{\hershey}{\fontfamily{hscs}\selectfont}
\newcommand{\tb}{\color{brown}}

\usetheme{Boadilla}
\usecolortheme{rose}

\newcommand\justify{\rightskip0pt \leftskip0pt}

%\usetheme{Goettingen}

\newenvironment{slide}
{\begin{frame}[environment=slide]
\frametitle{\insertsection \\ \insertsubsection}\justify\setlength{\parskip}{0.5cm}\vspace{-0.5cm}}
{\end{frame}}


\setbeamersize{text margin left=1cm}
\setbeamersize{text margin right=1cm}

%\setbeamersize{sidebar width left=1cm}
%\setbeamersize{sidebar width right=1cm}

\makeatletter
\newcommand{\setnextsection}[1]{%
  \setcounter{section}{\numexpr#1-1\relax}%
  \beamer@tocsectionnumber=\numexpr#1-1\relax\space}
\makeatother

\setbeamertemplate{itemize items}[triangle]
\setbeamercolor{itemize item}{fg=brown}
\setbeamertemplate{enumerate items}[circle]
\setbeamercolor{item projected}{bg=brown}
\setbeamercolor{block title}{fg=brown!100, bg=gray!20}
\setbeamercolor{block body}{bg=gray!10}

\title[Simulation]{Simulation of random variables}
\author[Jean-Michel Marin]{Jean-Michel Marin}

\institute[IMAG]{University of Montpellier \\
Faculty of Sciences}
\date[HAX918X]{HAX918X / 2024-2025}

\begin{document}

\frame{\titlepage}

\frame{\tableofcontents} 

\section{Methods involving the uniform distribution on $[0,1]$}

\begin{slide}

{\bf Proposition} Let $X$ be a real random variable ($X(\Omega)\subseteq \mathbb{R}$), 
with cumulative distribution function 
$F(x)=\P(X\leq x)=\int_{-\infty}^x f(u)d\mu(u)$
\begin{itemize}
\item If $F(x)$ is continuous, then $U=F(X)$ is distributed according to a uniform $[0,1]$ distribution 
\item Even if $F(x)$ is not continuous, the inequality $\P(F(X)\leq t)\leq t$ is true for all $t\in [0,1]$
\item If $F^{[-1]}(y)=\inf\{x:F(x)\geq y\}$ ($0<y<1$) and if $U$ is distributed from a uniform $[0,1]$ distribution, 
then $F^{[-1]}(U)$ is distributed according to $F(x)$ 
\end{itemize}

\end{slide}

\begin{slide}

To perform probabilistic simulations on a computer, 
a pseudo-random number generator is used

Such a generator returns a sequence $(x_n)_{n\in\mathbb{N}}$ of real numbers between 0 and 1
 
These numbers are calculated by a deterministic algorithm but imitate a realization of a 
sequence of iid uniform $[0,1]$ random variables

The good behavior of the sequence is verified by means of statistical tests

\end{slide}

\begin{slide}

A standard method to construct the sequence $(x_n)_{n\in\mathbb{N}}$
is the congruence $x_n=y_n/N$
where the $y_n$ are integers such that
$$
y_{n+1}=(ay_n+b)\quad\mbox{mod}\quad (N)
$$

The period of the congruence generator is always smallest 
than $N-1$

The choice of  $a$, $b$ et $N$ is done such that 
\begin{itemize}
\item the period of the generator is the largest as possible 
\item the sequence $(x_n)_{n\in\mathbb{N}}$ is as close as possible to an iid uniform 
$[0,1]$ sequence
\end{itemize}

\end{slide}

\begin{slide}

{\bf Proposition} If $U\sim\mathcal{U}([0,1])$ then $X=a+(b-a)U\sim\mathcal{U}([a,b])$

{\bf Proposition} If $U\sim\mathcal{U}([0,1])$ then $X=\mathbb{I}_{U\leq p}\sim\mathcal{B}(1,p)$ 

{\bf Proposition} If $U_1,\ldots,U_n$ are $n$ iid uniform random variables on $[0,1]$,
then $\ds X=\sum_{i=1}^n \mathbb{I}_{U_i\leq p}\sim\mathcal{B}(n,p)$

It is always possible to obtain a simulation following a random variable which takes the 
values $(x_i)_{i\in\mathbb{N}^*}$ with respective probabilities $(p_i)_{i\in\mathbb{N}^*}$ 
(with $p_i\geq 0$ such as $\sum_{i \in\mathbb{N}^*} p_i = 1$) using a single uniform 
variable on $[0,1]$

\end{slide}

\begin{slide}

{\bf Proposition} If $U\sim\mathcal{U}([0,1])$, then
$$
X=x_1\mathbb{I}_{U\leq p_1}+x_2\mathbb{I}_{p_1<U\leq p_1+p_2}+\ldots+
x_i\mathbb{I}_{\sum_{j=1}^{i-1}p_j<U\leq \sum_{j=1}^{i}p_j}+\ldots
$$
is distributes as a random variable that takes values $(x_i)_{i\in\mathbb{N}^*}$ 
with associates probabilities $(p_i)_{i\in\mathbb{N}^*}$

Requires coding a loop on $i$ with stopping rule \\ $\sum_{j=1}^{i-1}p_j<U\leq \sum_{j=1}^{i}p_j$ 
$\Longrightarrow$ it can take a while if the sequence $(p_i)_{i\in\mathbb{N}^*}$ converges slowly to 1.

\end{slide}

\begin{slide}

{\bf Proposition} If $U_1$ and $U_2$ are two $\mathcal{U}([0,1])$ 
independent random variables, then
$$
X_1=\sqrt{-2\ln(U_1)}\cos(2\pi U_2)
$$
and
$$
X_2=\sqrt{-2\ln(U_1)}\sin(2\pi U_2)
$$
are two independent standard Gaussian random variables

Recall that if $X\sim\mathcal{N}(0,1)$ then 
$\mu+\sigma X\sim\mathcal{N}(\mu,\sigma^2)$

\end{slide}

\section{The accept-reject algorithm}

\begin{slide}

{\bf Target} distribution with pdf $p$ on $\mathbb{R}^d$

{\bf Instrumental} distribution with pdf $q$ on $\mathbb{R}^d$ 

There exists $k\geq1$ such that
$$
\forall x\in\mathbb{R}^d,\quad p(x)\leq k q(x)
$$

\end{slide}

\begin{slide}

\begin{itemize}
\item[0)] Set i=1
\item[1)] Generate $Y_i$ from $q$
\item[2)] Calculate $\ds M=\frac{p(Y_i)}{kq(Y_i)}$
\item[3)] Generate $U_i\sim\mathcal{U}([0,1])$
\item[4)] If $U_i>M$, then $i=i+1$ and back $1)$ \\
If $U_i\leq M$, then $X=Y_i$
\end{itemize}

\end{slide}

\begin{slide}

Note $N=\inf\{i\geq 1: kq(Y_i)U_i\leq p(Y_i)\}$ ($N$ is a random variable),
we have $X=Y_N$

{\bf Proposition} $N$ is distributed according to a geometric distribution
with parameter $1/k$, $\E(N)=k$

$N$ is independent of $(Y_N,kq(Y_N)U_N)$ which is uniformly distributed on
$$
\{(x,z)\in\mathbb{R}^d\times \mathbb{R}:0\leq z\leq p(x)\}
$$

Typically, $X=Y_N$ is distributed from $p$

\end{slide}

\end{document}


